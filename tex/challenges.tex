%! TEX Root = ../proposal.tex

\section*{}
\subsection*{Research Challenges}

\begin{frame}{Challenges} %-----------------------------%

\begin{itemize}
    \item Optimal Trajectory Design
        \begin{itemize}
            \item Orbital dynamics are nonlinear and chaotic
            \item Very sensitive to initial conditions
            \item Intuition required by designer to enable convergence
        \end{itemize}
    \pause
    \item Transfers using low-thrust propulsion
        \begin{itemize}
            \item Requires long periods of thrusting/coasting
            \item Small perturbations require accurate numerical integration
            \item Difficult to capture the long-term effects accurately
        \end{itemize}
    \pause
    \item Direct Optimal Control
        \begin{itemize}
            \item Reformulate problem as parameter optimization
            \item Allows for use of nonlinear programming methods
            \item High dimensional problem and computationally intensive
            \item Results in suboptimal solutions due to discretization
        \end{itemize}
\end{itemize}
\end{frame}   %-----------------------------%

\begin{frame}[t]{Problem Formulation} %-----------------------------------%
\begin{itemize}
    \item \Emph{Constrained attitude control} : reorient vehicle while avoiding pointing at obstacles
    \begin{itemize}
        \item Exclusion zones for payloads e.g infrared telescope
        \item UAVs manuevering in congested locations
        \item Laser/Radio emitters on spacecraft
    \end{itemize}
    \pause
    \vs
    \item Previous approaches have several issues
    \begin{itemize}
        \item Attitude parameterizations: singularities/ambiguities
        \item Ad-hoc path planning: difficult to generalize to arbitrary obstacles
        \item Randomized methods: lack of stability guarantees
        \item Optimization based: expensive to compute and only provides open-loop control  
    \end{itemize}
\end{itemize}
\end{frame} %-------------------------------------%

\begin{frame}{Attitude Parameterizations}
    \begin{itemize}
        \item Euler Angles
        \begin{itemize}
            \item Minimal representation used for small attitude changes.
            \item Singularities exist for large angle slews: requires switching between 24 sequences
            \item Complicated trigonometric functions
        \end{itemize}
        \pause
        \vs
        \item Quaternion 
        \begin{itemize}
            \item No singularities
            \item Two anti-podal quaternions for the same attitude
            \item Unwinding behavior for control systems
        \end{itemize}
        \pause
        \vs
        \item Geometric control
        \begin{itemize}
            \item Globally and uniquely characterize attitude: \( R \in \SO \)
            \item Controller is globally valid for large angle maneuvers
        \end{itemize}
    \end{itemize}
    
\end{frame}