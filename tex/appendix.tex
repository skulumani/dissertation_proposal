% !TEX Root = ../proposal.tex

\appendix
\section*{}
\subsection*{Appendix}

\begin{frame}[noframenumbering,label=grav] %---------------------------------------%
\frametitle{Gravity Expansion}
\only<1>{
\begin{block}{Force due to gravity on body \( B\) from particle \( P \)}
\begin{equation*}
    F = -G m_P \int_{\mathcal{B}} \bar{r} \parenth{r^2}^{-\frac{3}{2}} \, d m
\end{equation*}
\end{block}
}
\only<2>{
\begin{block}{Binomial Expansion}

\begin{align*}
    F &= - \frac{G m_P m_B}{R^2} \parenth{\hat{a}_1 + \sum_{i=2}^\infty \bar{f}^{(i)}} \\
    \bar{f}^{(2)} &= \frac{1}{m_B R^2} \braces{\frac{3}{2} \bracket{ \tr{J_B} - 5 \hat{a}_1 \cdot J_B \cdot \hat{a}_1} \hat{a}_1 + 3 J_B \cdot \hat{a}_1}
\end{align*}
\end{block}
}
\only<3>{
\begin{block}{Gravity Moment}
\begin{equation*}
    M = \frac{3 G m_B}{R^3} \hat{a}_1 \times J \cdot \hat{a}_1 + \frac{G m_B m_P}{R} \sum_{i=3}^\infty \hat{a}_1 \times \bar{f}^{(i)}
\end{equation*}
%\begin{center}
%CG \( \neq \) CM 
%\end{center}
\end{block}
}
\hyperlink{potential}{\beamerreturnbutton{Go back}}
\end{frame}%-------------------------------------------------------------%


\begin{frame}[noframenumbering]{Solar Radiation Pressure} %---------------------------------------------------%

\begin{block}{Constant Area Approximation}
Momentum transfer from solar photons striking spacecraft
\begin{align*}
    a_{SRP} = - \frac{\parenth{1 + \rho} P_0 A_{S}}{M_S} \frac{d - r}{\norm{d-r}^3}
\end{align*}

\( P_0\) is a solar flux constant \SI{1e8}{\kilogram\kilo\meter\cubed\per\second\squared\per\meter\squared}

\( B_S = \frac{M_S}{A_S} \) is a mass to area ratio - \( 20 - 40 \, \si{\kilogram\per\meter\squared} \)

\( \rho \) is total reflectance or albedo of body

\( d, r \) defined in small body frame to Sun and S/C respectively
\end{block}
\begin{itemize}
    \item Large bodies, i.e. 433 Eros, SRP may be neglected
    \item Small bodies, \( < 1-5 \si{\kilo\meter} \), SRP is crucial
\end{itemize}
\end{frame} %-----------------------------------------------------------%

\begin{frame}[noframenumbering,label=astro] %----------------------------------------------%
\frametitle{Astrodynamics}

\only<1>{
\begin{block}{Newton's Law of Universal Gravitation}
    Any two bodies attract one another with a force proportional to the product of their masses and invesely proportional to the square of the distance between them
    \[
    F = - \frac{G m_1 m_2}{ r^2} \frac{\bar{r}}{r}
    \]
\end{block}
}
\only<2>{
\begin{block}{N-Body}
    Gravitational attraction of \( n \) bodies acting on the particle of interest \( m_i \)
    \[
    m_i \ddot{\bar{r}}_i = -G \sum_{\substack{j = 1\\j\neq i}}^n \frac{m_i m_j}{r_{ji}^3} \bar{r}_{ji}
    \]
    Motion of \( \bar{r}_j (t) \) is not known - Not solvable in general
    
\end{block}
}
\hyperlink{sim}{\beamerreturnbutton{Go back}}
\end{frame} %--------------------------------------------------------------%


\begin{frame}[noframenumbering]{Astrodynamics}%-------------------------------------------------------%
\only<1>{
\begin{block}{Integrals of Motion}
    Require \( 6 n \) integrals of motion but know \( 10 \)
        \begin{itemize}
            \item Linear Momentum - system CM constant speed - \( 6 \) constants
            \item Angular Momentum - system angular moment - \( 3 \) constants
            \item Total Energy - Conservative system - \( 1 \) constant
        \end{itemize}
    Even two-body problem is not solvable
\end{block}
}
\only<2>{
\begin{block}{Relative Motion}
Really care about relative motion of \( m_i \) wrt \( m_q \)
\[
    \ddot{\bar{r}}_{qi} + G \frac{m_i + m_q}{r_{qi}^3} \bar{r}_{qi} = G \sum_{\substack{j = 1\\j\neq i,q}}^n m_j \parenth{ \frac{\bar{r}_{ij}}{r_{ij}^3} - \frac{\bar{r}_{qj}}{r_{qj}^3}} \bar{r}_{ji}
\]
\end{block}
}
\hyperlink{sim}{\beamerreturnbutton{Go back}}
\end{frame} %----------------------------------------------------------%



\begin{frame}[noframenumbering]{Lagrange's Equations}%---------------------------------------------%
%\begin{block}{Conservative System}
%   All applied forces \( F_i \) are derivable from a potential function \( V(x_1, x_2, \ldots, x_{3N}) \) : \(F_i = -\deriv{V}{x_i}\)
%\end{block}
\only<1>{
\begin{block}{Hamilton's Principle}
    The actual path in configuration space followed by a holonomic system during the fixed interval \( t_0 \) to \( t_1\) is such that the action integral:
    \[
    S = \int_{t_0}^{t_1} L(q, \dot{q}) dt
    \]
    is stationary with respect to path variations which vanish at the end-points.
\end{block}
}
\only<2>{
\begin{block}{Variation of action integral}
\begin{align*}
    \delta S &= \int_{t_0}^{t_1} \deriv{L}{q} \delta q + \deriv{L}{\dot{q}} \delta \dot{q} \, dt \\
        &= \int_{t_0}^{t_1} \deriv{L}{q} \delta q - \frac{d}{dt} \left( \deriv{L}{\dot{q}}\right) \delta q \, dt - \left. \left[ \deriv{L}{\dot{q}} \delta q\right] \right|_0^T \\
    &= \int_{t_0}^{t_1} \deriv{L}{q} - \frac{d}{dt} \left( \deriv{L}{\dot{q}}   \right) \, dt \, ,
\end{align*}
\end{block}
}
\end{frame}%---------------------------------------------------------------------------%

\begin{frame}[noframenumbering,label=jacobi]{Jacobi Integral}%-------------------------------------------------------------%
\only<1>{
\begin{block}{Total Mechanical Energy}
\begin{enumerate}
    \item Generalized force from potential function: \( Q_i = -\deriv{V}{q_i} \)
    \item Work is path independent: \( W = \sum_{i=1}^{n} \int_{A_i}^{B_i} Q_i \, dq_i \)
\end{enumerate}

    If no other forces do work, then total mechanical energy is conserved:  \(  E(q, \dot{q}) = T + V   \)
\end{block}
}
\only<2>{
\begin{block}{More general constant}
\begin{enumerate}
    \item Standard Form of Lagrange's equation applies
    \item Lagrangian is not explicit fcn of time
    \item Constraints may be expressed: \( \sum_{i=1}^n a_{ji} d q_i = 0\)
\end{enumerate}

\[
    h = \sum_{i=1}^n \deriv{L}{\dot{q}_i} \dot{q}_i - L 
\]
\end{block}
}
\hyperlink{zvc}{\beamerreturnbutton{Go back}}
\end{frame} %-----------------------------------------------------------%

\begin{frame}[noframenumbering] %------------------------------------%
\frametitle{Variational Principle}
    \begin{itemize}
        \item Variational Integrators
            \begin{itemize}
                \item Structure-preserving integrators for Hamiltonian systems
                \item Obtained by discretizing variational principle
            \end{itemize}
    \end{itemize}
    \pause
    \begin{columns}[c]
        \begin{column}{0.5\textwidth}
            \centering
            \begin{beamercolorbox}[wd=0.8\columnwidth,sep=0.05cm,center]{numerical} Continuous Time \end{beamercolorbox}
            \begin{beamercolorbox}[wd=0.8\columnwidth,sep=0.05cm,center]{numerical} 
                Configuration Space \\
                \( \parenth{q, \dot{q} } \in TQ \)
            \end{beamercolorbox}
            \begin{beamercolorbox}[wd=0.8\columnwidth,sep=0.05cm,center]{numerical} 
                Lagrangian \\
                \( L\parenth{q, \dot{q} } \)
            \end{beamercolorbox}
            \begin{beamercolorbox}[wd=0.8\columnwidth,sep=0.05cm,center]{numerical} 
                Action Integral \\
                \( S = \int_{0}^T L\left( q, \dot{q}\right) \, dt \)
            \end{beamercolorbox}
            \begin{beamercolorbox}[wd=0.8\columnwidth,sep=0.05cm,center]{numerical} 
                Stationary Action \\
                \( \delta S = 0 \)
            \end{beamercolorbox}
%           \begin{beamercolorbox}[wd=0.8\columnwidth,sep=0.05cm,center]{numerical} 
%               Legendre Transform \\
%               \( p_i = \deriv{L}{\dot{q}} \)
%           \end{beamercolorbox}
            \begin{beamercolorbox}[wd=0.8\columnwidth,sep=0.05cm,center]{numerical} 
                Equation of Motion \\
                \( \ddot{q} = f \parenth{q, \dot{q} } \)
            \end{beamercolorbox}
        \end{column}
        \pause
        \begin{column}{0.5\textwidth}
            \centering
            \begin{beamercolorbox}[wd=0.8\columnwidth,sep=0.05cm,center]{numerical} Discrete Time \end{beamercolorbox}
            \begin{beamercolorbox}[wd=0.8\columnwidth,sep=0.05cm,center]{numerical} 
                Configuration Space \\
                \( \parenth{q_k, q_{k+1} } \in Q \times Q \)
            \end{beamercolorbox}
            \begin{beamercolorbox}[wd=0.8\columnwidth,sep=0.05cm,center]{numerical} 
                Lagrangian \\
                \( L_d\parenth{q_k, q_{k+1}} \)
            \end{beamercolorbox}
            \begin{beamercolorbox}[wd=0.8\columnwidth,sep=0.05cm,center]{numerical} 
                Action Sum \\
                \( S_d = \sum_{k=0}^{N-1} L_d(q_k, q_{k+1}) \)
            \end{beamercolorbox}
            \begin{beamercolorbox}[wd=0.8\columnwidth,sep=0.05cm,center]{numerical} 
                Stationary Action \\
                \( \delta S_d = 0 \)
            \end{beamercolorbox}
%           \begin{beamercolorbox}[wd=0.8\columnwidth,sep=0.05cm,center]{numerical} 
%               Fiber Derivative \\
%               \( p_k = - \deriv{L_d(q_k, q_{k+1})}{q_k} \) \\
%               \( p_{k+1} = \deriv{L_d(q_k, q_{k+1})}{q_{k+1}} \)
%           \end{beamercolorbox}
            \begin{beamercolorbox}[wd=0.8\columnwidth,sep=0.05cm,center]{numerical} 
                Equation of Motion \\
                \( q_{k+2} = f_d \parenth{q_k, q_{k+1} } \)
            \end{beamercolorbox}
        \end{column}
    \end{columns}
    
    \note[itemize]{
        \item Continous time - discretization occurs at end while implemented in digital computer
        \item Discrete time - discretization occurs at the beginning. 
        Dynamics derived in discrete time
        \item Legendre transform allows for expression of dynamics in Hamiltonian form
    }
\end{frame}%-----------------------------------------%

\begin{frame}[noframenumbering]{Dynamics of Rigid Body} %------------------------------------------------%
    \begin{block}{Newton's Law}
        
        \begin{align*}
            F_x &= m \parenth{\dot{v}_x + v_z \omega_y - v_y \omega_z} \\
            F_y &= m \parenth{\dot{v}_y + v_x \omega_z - v_z \omega_x} \\
            F_z &= m \parenth{\dot{v}_z + v_y \omega_x - v_x \omega_y} \\
        \end{align*}
    \end{block}
    
    \begin{block}{Euler's Law}
        
        \begin{align*}
            {M}_x &= I_{xx} \dot{\omega}_x + \parenth{I_{zz}-I_{yy}} \omega_y \omega_z \\
            {M}_y &= I_{yy} \dot{\omega}_y + \parenth{I_{xx}-I_{zz}} \omega_z \omega_x \\
            {M}_z &= I_{zz} \dot{\omega}_z + \parenth{I_{yy}-I_{xx}} \omega_x \omega_y \\
        \end{align*}
    \end{block}
\end{frame} %----------------------------------------------------------------------%

\begin{frame}[noframenumbering]{Spacecraft Propulsion}%------------------------------------------------%
\begin{block}{Ideal Rocket Equation}
    Amount of propellant required for a given velocity change
    \[
        \Delta V = - v_e \ln\parenth{\frac{m_f}{m_i}}
    \]
\end{block}

High exhaust speeds make electric propulsion attractive

Much higher efficiency than chemical propulsion
\end{frame} %----------------------------------------------------------------%


\begin{frame}{Attitude Parameterizations}
    \begin{itemize}
        \item Euler Angles
        \begin{itemize}
            \item Minimal representation used for small attitude changes.
            \item Singularities exist for large angle slews: requires switching between 24 sequences
            \item Complicated trigonometric functions
        \end{itemize}
        \pause
        \vs
        \item Quaternion 
        \begin{itemize}
            \item No singularities
            \item Two anti-podal quaternions for the same attitude
            \item Unwinding behavior for control systems
        \end{itemize}
        \pause
        \vs
        \item Geometric control
        \begin{itemize}
            \item Globally and uniquely characterize attitude: \( R \in \SO \)
            \item Controller is globally valid for large angle maneuvers
        \end{itemize}
    \end{itemize}

\end{frame}

\begin{frame}[t]\frametitle{Constrained Attitude Control}
    \begin{itemize}
    \item \Emph{Constrained attitude control} : reorient vehicle while avoiding pointing at obstacles
    \begin{itemize}
        \item Exclusion zones for payloads e.g infrared telescope
        \item UAVs manuevering in congested locations
        \item Laser/Radio emitters on spacecraft
    \end{itemize}
    \pause
    \vs
    \item Previous approaches have several issues
    \begin{itemize}
        \item Attitude parameterizations: singularities/ambiguities
        \item Ad-hoc path planning: difficult to generalize to arbitrary obstacles
        \item Randomized methods: lack of stability guarantees
        \item Optimization based: expensive to compute and only provides open-loop control  
    \end{itemize}
\end{itemize}


\end{frame}