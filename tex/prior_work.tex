% !TEX Root = ../proposal.tex

\section*{Prior Research}
\subsection*{Reachability Set for Orbital Transfers}

\begin{frame}{Orbital Transfers} % -----------------------------------%
    \begin{itemize}
        \item \Emph{Reachability set} allows for systematic transfer design
        \begin{itemize}
            \item Transfer design on lower dimensional \Poincare surface
            \item Simple method to incorporate effects of low-thrust 
            \item Avoids the issue of determining initial conditions
        \end{itemize}
    \pause
        \item Alleviates many issues with previous approaches
        \begin{itemize}
            \item Initial states chosen from from the reachable set
            \item Indirect optimal control vs. direct optimal control
            \item Reachability set gives bounds on motion
        \end{itemize}    
    \end{itemize}

  \note[itemize]{
    \item Reachability set avoids the need to pick initial conditions
    \item We compute on a lower dimensional surface
  }
\end{frame} %--------------------------------------%

\subsection*{Dynamic Systems Theory}

\begin{frame}{\Poincare map}
\begin{itemize}
    \item Intersection of a periodic orbit with a lower dimensional subspace,
    \pause
        \begin{itemize}
            \item  \Emph{\Poincare section} - discrete map between intersections
        \end{itemize}
        \pause
    \item Useful for investigating the stability and structure 
    \pause
    \item Define a \Poincare section \( \Sigma \) 
        \begin{itemize}
            \item Used for initial and target periodic orbits
            \item Subspace for the \Emph{reachability set}
        \end{itemize}
\end{itemize}

\begin{align*}
    \Sigma = \braces{\parenth{x, \dot{x}, z, \dot{z}} | y(t_f) = 0 }
\end{align*}

{\LARGE Add image of \Poincare section}
\end{frame}

\begin{frame}{Reachability Set}

\begin{itemize}
    \item Set of states achievable from a given initial condition over fixed \( t_f \) s.t. maximum control constraint
    \[
        R( \vecbf{x}_0, \mathcal{U} , t_f) = \braces{ \vecbf{x}_f \subseteq \mathcal{X} | \exists \vecbf{u} \in \mathcal{U}, \vecbf{x}(t_f) = \vecbf{x}_f }
    \]
    \pause
    \item Directly derivable from optimal control
    \item Frequently used for safety planning, e.g. air traffic avoidance
    \pause
    \item Extend to the design of orbital transfers
\end{itemize}

\end{frame}

\begin{frame}{Reachability Set on \Poincare section} % -----------------------------------%

\begin{itemize}
    \item Generate the reachability set on a \Poincare section
    \[
        \Sigma = \braces{\parenth{x, \dot{x}, z, \dot{z}} | y(t_f) = 0 }
    \]
    \item Control input is chosen to enlarge the reachable set
\end{itemize}
\pause
\begin{center}
    \begin{scaletikzpicturetowidth}{0.4\textwidth}
    \begin{tikzpicture}[scale=\tikzscale]
        \coordinate [label=left:\textcolor{black}{\large \(\vecbf{x}_0\)}] (x0) at (-1,-2);
        \coordinate [label=below:\textcolor{black}{\large  \(\vecbf{x}_n\)}] (xn) at (1,1);
        \coordinate [label=left:\textcolor{black}{\large  \(\Sigma\)}] (sigma) at (-4,3);
        %\coordinate [label=below:\textcolor{black}{\large  \(P(\vec{x})\)}] (P) at (0,-3.5);
        % define the path of the flow with coordinates
        \coordinate [label=right:\textcolor{black}{}] (f1) at (5,-2);
        \coordinate [label=below:\textcolor{black}{\large  \(\psi(t,\vecbf{x}_0)\)}] (f2) at (2,-5);
        \coordinate [label=right:\textcolor{black}{}] (f3) at (-4,-4);
        \coordinate [label=right:\textcolor{black}{}] (f4) at (-4,-1);
        
    %   \draw[help lines] (-10,-10) grid (10,10); %grid
        \filldraw [black] (x0) circle [radius=3pt];
        \filldraw [black] (xn) circle [radius=3pt];
    
        \draw [ultra thick,black,->-](x0) to[out=20,in=90,distance=2cm] (f1) to[out=-90,in=0,distance=2cm] (f2) to[out=180,in=-45,distance=2cm] (f3) to[out=135,in=-135,distance=2cm] (f4) ;
        \draw [ultra thick, black,dashed,->] (f4) to[out=45,in=180,distance=1cm] ($(xn)-(2,0)$);
        
        \draw [ultra thick] plot [smooth cycle, tension=0.1, rotate=5] coordinates { (-4,-3) (4,-3) (4,3) (-4,3) };
    
        \draw [thick,dashed] (xn) circle [radius=2cm]; % reachability set
    
        \draw [thick,->] (xn) -- ($(xn) + (2.5,0)$);
        \draw [thick,rotate=45,->] (xn) -- ($(xn) + (2.5,0)$);
        \draw ($(xn) + (1,0)$) arc [start angle=0,end angle=45, radius=1];
        \node [draw=none] at (2.8,1.5) {\large \(\phi_d\)};
        \draw [decorate,decoration={brace,amplitude=5pt},rotate=45] (xn) -- ($(xn) + (2,0)$);
        \node [draw=none] at ($ (xn) + (0,1) $) {\large \( J \)};
    \end{tikzpicture}
    \end{scaletikzpicturetowidth}
\end{center}

{\LARGE IMPROVE THIS TIKZ IMAGE}
\end{frame} %--------------------------------------%

\subsection*{Optimal Control Problem}

\begin{frame}{Optimal Control Problem}
\begin{itemize}
    \item Reachability defined as distance between controlled and uncontrolled states
        \[
            J = -\frac{1}{2} \left( \vecbf{x}(t_f) - \vecbf{x}_{n}(t_f)\right)^T 
            Q
            \left( \vecbf{x}(t_f) - \vecbf{x}_{n}(t_f)\right) 
        \]
    
    \pause
    \item Terminal constraints - \( \vecbf{m}_i(\vecbf{x}_f) = 0\) ensures intersection with \( \Sigma \)
    \pause
    \item Control constraint used to emulate realistic system
        \[
            c(\vecbf{u}) = \vecbf{u}^T \vecbf{u} - u_m^2 \leq 0 
        \]
\end{itemize}

\end{frame}

\begin{frame}{Solving the Optimal Control Problem}
\begin{itemize}
    \item Multiple shooting used to solve necessary conditions
    \pause
    \item Approximate the reachable set via \( \phi_i \) 
    \begin{itemize}
        \item Parameterize a direction on \( \Sigma \)
    \end{itemize}
    \pause
    \item From the reachable set we chose the state which minimizes \( d \) 
    \item Compute another reachable set if target is not feasible
\end{itemize}
\[
    d = \norm{\vecbf{x}_f - \vecbf{x}_t} 
\]

{\LARGE Multiple shooting diagram}
\end{frame}

% results from 2015 AAS
\subsection*{Transfer Examples}

\begin{frame}[t]{Orbital Transfers via Reachability Sets}
    \begin{itemize}
        \item Numerical simulations in two different environments
        \begin{itemize}
            \item Planar Circular Restricted Three Body Problem
            \item Restricted Two Body Problem 
        \end{itemize}
        \pause
        \item Dynamics are related but vary in complexity
        \begin{itemize}
            \item Planar vs. Three Dimensional
            \item Gravitational Potential
            \item Both defined in a rotating reference frame
        \end{itemize}
    \end{itemize}
\end{frame}

\subsubsection{Three Body Problem}

\begin{frame}%--------------------------------------------%
\frametitle{Three Body Problem}
    \begin{itemize}
        \item Transfer from \( L_1 \) orbit to periodic orbit near the Moon
        \item Bounded control input and fixed time horizon
    \end{itemize}
    \visible<2>{
        \begin{center}
            \includegraphics[width=0.7\textwidth,height=0.7\textheight,keepaspectratio]{2015AAS/moon_orbit.pdf}
        \end{center}
        }
    
    \note[itemize]{
        \item Introduce problem
        \item Possible use as a communication array
        \item Might actually be a distant retrograde orbit
        }
\end{frame} %--------------------------------------------%

\begin{frame}%------------------------------------------------%
    \frametitle{Reachable Set Transfer}
    \begin{itemize}
        \item Approximate the reachable set on the \Poincare section
        \begin{itemize}
            \item Generate many optimal solutions
        \end{itemize}
        \item<3-> Intersection point used to generate a transfer
        \begin{itemize}
            \item Shorter time of flight than uncontrolled dynamics
        \end{itemize}
    \end{itemize}
    \begin{center}
        \only<1-2>{\includegraphics[width=0.5\textwidth,height=0.7\textheight,keepaspectratio]{2015AAS/reach_trajectory}~}
        \only<3->{\includegraphics[width=0.5\textwidth,height=0.7\textheight,keepaspectratio]{2015AAS/reach_transfer}~}
        \only<2-3>{\includegraphics[width=0.5\textwidth,height=0.7\textheight,keepaspectratio]{2015AAS/poincare_compare}} 
    \end{center}
    
    \note[itemize]{
        \item Compare to reachable set approach
        \item Shorter time of flight
        \item Multiple shooting to solve TPBVP
        Vary \( \theta\) to change direction on section
        Linear interpolation to determine intersection on Poincar\'e section
        }
\end{frame} %--------------------------------------------------%

\begin{frame}{Geostationary  transfer} %----------------------------------------------------%
    \begin{itemize}
           \item Tranfer for geostationary orbit to a periodic orbit about \( L_1\)
           \item Multiple iterations of reachable set required for transfer
    \end{itemize}

    \begin{center}
       \only<1>{
       \includegraphics[width=0.5\textwidth,height=0.7\textheight,keepaspectratio]{figures/2015ACTA/initial_final.pdf}~
       \includegraphics[width=0.5\textwidth,height=0.7\textheight,keepaspectratio]{figures/2015ACTA/geo_transfer_full.pdf} 
       }

       \only<2>{
       \includegraphics[width=0.5\textwidth,height=0.7\textheight,keepaspectratio]{figures/2015ACTA/geo_transfer_zoom.pdf}~
       \includegraphics[width=0.5\textwidth,height=0.7\textheight,keepaspectratio]{figures/2015ACTA/poincare.pdf}
       }
    \end{center}

\end{frame} %--------------------------------------------%

\subsubsection*{Asteroid transfers}

% Results from 2016 AAS
\begin{frame}{Transfer Objective} %-----------------------------%

\begin{itemize}
    \item Goal is to transfer between two equatorial periodic orbits
    \item Typical scenario during study of an asteroid
\end{itemize}

\begin{center}
    \includegraphics[width=0.5\textwidth,height=0.7\textheight,keepaspectratio]{figures/2016AAS/initial_transfer.pdf}~
    \includegraphics[width=0.5\textwidth,height=0.7\textheight,keepaspectratio]{figures/2016AAS/initial_transfer_3d.pdf}
\end{center}

\end{frame}%-----------------------------%

\begin{frame}{Simulation}

\begin{itemize}
    \item Generate the reachability set through discretization of \( \phi_i \)
    \item Visualize \( \Sigma \in \R^4 \) through the use of two 2-D sections
    \pause
    \item Control input allows for large deviation in velocity components
\end{itemize}

\begin{center}
    \includegraphics[width=0.5\textwidth,height=0.7\textheight,keepaspectratio]{figures/2016AAS/poincare_xvsxdot.pdf}~
    \includegraphics[width=0.5\textwidth,height=0.7\textheight,keepaspectratio]{figures/2016AAS/poincare_zvszdot.pdf}
\end{center}

\end{frame}

\begin{frame}{Transfer Simulation}
    \begin{itemize}
        \item Four iterations of the reachable state to meet the target set
        \item Final transfer is computed with a fixed terminal state constraint
    \end{itemize}

    \begin{center}
        \includegraphics[width=0.5\textwidth,height=0.7\textheight,keepaspectratio]{figures/2016AAS/trajectory.pdf}~
        \includegraphics[width=0.5\textwidth,height=0.7\textheight,keepaspectratio]{figures/2016AAS/trajectory_3d.pdf}
    \end{center}

\end{frame}

\begin{frame}{Complete transfer}
\begin{itemize}
    \item We can visualize the complete trajectory in both the body and inertial frames
\end{itemize}

\begin{center}
  \animategraphics[draft,autoplay,loop,width=0.5\textwidth,height=0.7\textheight,keepaspectratio]{30}{animation/2016AAS/body/IMG}{00001}{01499}~\hfill
  \animategraphics[draft,autoplay,loop,width=0.5\textwidth,height=0.7\textheight,keepaspectratio]{30}{animation/2016AAS/inertial/IMG}{00001}{01499}
\end{center}

\end{frame}

