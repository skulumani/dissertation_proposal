% !TEX Root = ../proposal.tex

\section*{Research Description}
\subsection*{Thesis}

% THESIS STATEMENT - entire dissertation should support and defend this statement
\begin{frame}[t]{Proposed research}
    \begin{itemize}
        \item Geometric mechanics enables the novel use low-thrust propulsion for operations about asteroids
        \begin{itemize}     
            \item Exploit the natural dynamics through \Poincare sections and reachability
            \item Variational integrators for long duration accuracy
            \item Accurate system models using \( \SE \)
            \item Globally stable attitude controllers in the presence of obstacles
        \end{itemize}
    \end{itemize}

\end{frame}

\begin{frame}[t]{Proposed Approach} % -----------------------------------%
  \begin{itemize}
      \item \Emph{Reachability set} on \Poincare section allows for systematic transfer design
        \begin{itemize}
            \item Transfer design on lower dimensional subspace
            \item Simple method to incorporate effects of low-thrust 
            \item Avoids the issue of determining initial conditions
        \end{itemize}
        \pause
      \item Extension of previous work in planar three-body problem     
  \end{itemize}

  \note[itemize]{
    \item Reachability set avoids the need to pick initial conditions
    \item We compute on a lower dimensional surface
  }
\end{frame} %--------------------------------------%

\begin{frame}[t]{Objective} %---------------------------------------%

  \begin{block}{Nonlinear Control Design}
    Design control input \( u \) that stabilizes system from initial attitude \( R_0 \) to desired attitude \( R_d \) while avoiding obstacles
  \end{block}
  \pause
  \vs
  \begin{itemize}
    \item Avoid drawbacks of other approaches 
    \begin{itemize}
      \item \Emph{Geometric control} - analysis is conducted directly on \( \SO \) 
      \item \Emph{Barrier function} - allows for arbitrary amount of constraints
      \item \Emph{Efficient } - real time feedback control
      \item \Emph{Stability} - Lyapunov analysis gives rigourous stability proof
      \item \Emph{Adaptive} - handles system uncertainties
    \end{itemize}
  \end{itemize}
\end{frame}